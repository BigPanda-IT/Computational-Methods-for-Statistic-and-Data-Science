% Options for packages loaded elsewhere
\PassOptionsToPackage{unicode}{hyperref}
\PassOptionsToPackage{hyphens}{url}
%
\documentclass[
]{article}
\usepackage{amsmath,amssymb}
\usepackage{iftex}
\ifPDFTeX
  \usepackage[T1]{fontenc}
  \usepackage[utf8]{inputenc}
  \usepackage{textcomp} % provide euro and other symbols
\else % if luatex or xetex
  \usepackage{unicode-math} % this also loads fontspec
  \defaultfontfeatures{Scale=MatchLowercase}
  \defaultfontfeatures[\rmfamily]{Ligatures=TeX,Scale=1}
\fi
\usepackage{lmodern}
\ifPDFTeX\else
  % xetex/luatex font selection
\fi
% Use upquote if available, for straight quotes in verbatim environments
\IfFileExists{upquote.sty}{\usepackage{upquote}}{}
\IfFileExists{microtype.sty}{% use microtype if available
  \usepackage[]{microtype}
  \UseMicrotypeSet[protrusion]{basicmath} % disable protrusion for tt fonts
}{}
\makeatletter
\@ifundefined{KOMAClassName}{% if non-KOMA class
  \IfFileExists{parskip.sty}{%
    \usepackage{parskip}
  }{% else
    \setlength{\parindent}{0pt}
    \setlength{\parskip}{6pt plus 2pt minus 1pt}}
}{% if KOMA class
  \KOMAoptions{parskip=half}}
\makeatother
\usepackage{xcolor}
\usepackage[margin=1in]{geometry}
\usepackage{color}
\usepackage{fancyvrb}
\newcommand{\VerbBar}{|}
\newcommand{\VERB}{\Verb[commandchars=\\\{\}]}
\DefineVerbatimEnvironment{Highlighting}{Verbatim}{commandchars=\\\{\}}
% Add ',fontsize=\small' for more characters per line
\usepackage{framed}
\definecolor{shadecolor}{RGB}{248,248,248}
\newenvironment{Shaded}{\begin{snugshade}}{\end{snugshade}}
\newcommand{\AlertTok}[1]{\textcolor[rgb]{0.94,0.16,0.16}{#1}}
\newcommand{\AnnotationTok}[1]{\textcolor[rgb]{0.56,0.35,0.01}{\textbf{\textit{#1}}}}
\newcommand{\AttributeTok}[1]{\textcolor[rgb]{0.13,0.29,0.53}{#1}}
\newcommand{\BaseNTok}[1]{\textcolor[rgb]{0.00,0.00,0.81}{#1}}
\newcommand{\BuiltInTok}[1]{#1}
\newcommand{\CharTok}[1]{\textcolor[rgb]{0.31,0.60,0.02}{#1}}
\newcommand{\CommentTok}[1]{\textcolor[rgb]{0.56,0.35,0.01}{\textit{#1}}}
\newcommand{\CommentVarTok}[1]{\textcolor[rgb]{0.56,0.35,0.01}{\textbf{\textit{#1}}}}
\newcommand{\ConstantTok}[1]{\textcolor[rgb]{0.56,0.35,0.01}{#1}}
\newcommand{\ControlFlowTok}[1]{\textcolor[rgb]{0.13,0.29,0.53}{\textbf{#1}}}
\newcommand{\DataTypeTok}[1]{\textcolor[rgb]{0.13,0.29,0.53}{#1}}
\newcommand{\DecValTok}[1]{\textcolor[rgb]{0.00,0.00,0.81}{#1}}
\newcommand{\DocumentationTok}[1]{\textcolor[rgb]{0.56,0.35,0.01}{\textbf{\textit{#1}}}}
\newcommand{\ErrorTok}[1]{\textcolor[rgb]{0.64,0.00,0.00}{\textbf{#1}}}
\newcommand{\ExtensionTok}[1]{#1}
\newcommand{\FloatTok}[1]{\textcolor[rgb]{0.00,0.00,0.81}{#1}}
\newcommand{\FunctionTok}[1]{\textcolor[rgb]{0.13,0.29,0.53}{\textbf{#1}}}
\newcommand{\ImportTok}[1]{#1}
\newcommand{\InformationTok}[1]{\textcolor[rgb]{0.56,0.35,0.01}{\textbf{\textit{#1}}}}
\newcommand{\KeywordTok}[1]{\textcolor[rgb]{0.13,0.29,0.53}{\textbf{#1}}}
\newcommand{\NormalTok}[1]{#1}
\newcommand{\OperatorTok}[1]{\textcolor[rgb]{0.81,0.36,0.00}{\textbf{#1}}}
\newcommand{\OtherTok}[1]{\textcolor[rgb]{0.56,0.35,0.01}{#1}}
\newcommand{\PreprocessorTok}[1]{\textcolor[rgb]{0.56,0.35,0.01}{\textit{#1}}}
\newcommand{\RegionMarkerTok}[1]{#1}
\newcommand{\SpecialCharTok}[1]{\textcolor[rgb]{0.81,0.36,0.00}{\textbf{#1}}}
\newcommand{\SpecialStringTok}[1]{\textcolor[rgb]{0.31,0.60,0.02}{#1}}
\newcommand{\StringTok}[1]{\textcolor[rgb]{0.31,0.60,0.02}{#1}}
\newcommand{\VariableTok}[1]{\textcolor[rgb]{0.00,0.00,0.00}{#1}}
\newcommand{\VerbatimStringTok}[1]{\textcolor[rgb]{0.31,0.60,0.02}{#1}}
\newcommand{\WarningTok}[1]{\textcolor[rgb]{0.56,0.35,0.01}{\textbf{\textit{#1}}}}
\usepackage{graphicx}
\makeatletter
\def\maxwidth{\ifdim\Gin@nat@width>\linewidth\linewidth\else\Gin@nat@width\fi}
\def\maxheight{\ifdim\Gin@nat@height>\textheight\textheight\else\Gin@nat@height\fi}
\makeatother
% Scale images if necessary, so that they will not overflow the page
% margins by default, and it is still possible to overwrite the defaults
% using explicit options in \includegraphics[width, height, ...]{}
\setkeys{Gin}{width=\maxwidth,height=\maxheight,keepaspectratio}
% Set default figure placement to htbp
\makeatletter
\def\fps@figure{htbp}
\makeatother
\setlength{\emergencystretch}{3em} % prevent overfull lines
\providecommand{\tightlist}{%
  \setlength{\itemsep}{0pt}\setlength{\parskip}{0pt}}
\setcounter{secnumdepth}{-\maxdimen} % remove section numbering
\ifLuaTeX
  \usepackage{selnolig}  % disable illegal ligatures
\fi
\usepackage{bookmark}
\IfFileExists{xurl.sty}{\usepackage{xurl}}{} % add URL line breaks if available
\urlstyle{same}
\hypersetup{
  pdftitle={Project},
  hidelinks,
  pdfcreator={LaTeX via pandoc}}

\title{Project}
\author{}
\date{\vspace{-2.5em}2024-12-05}

\begin{document}
\maketitle

\begin{Shaded}
\begin{Highlighting}[]
\CommentTok{\# Load necessary libraries}
\FunctionTok{library}\NormalTok{(tidyverse)}
\end{Highlighting}
\end{Shaded}

\begin{verbatim}
## -- Attaching core tidyverse packages ------------------------ tidyverse 2.0.0 --
## v dplyr     1.1.4     v readr     2.1.5
## v forcats   1.0.0     v stringr   1.5.1
## v ggplot2   3.5.1     v tibble    3.2.1
## v lubridate 1.9.3     v tidyr     1.3.1
## v purrr     1.0.2     
## -- Conflicts ------------------------------------------ tidyverse_conflicts() --
## x dplyr::filter() masks stats::filter()
## x dplyr::lag()    masks stats::lag()
## i Use the conflicted package (<http://conflicted.r-lib.org/>) to force all conflicts to become errors
\end{verbatim}

\begin{Shaded}
\begin{Highlighting}[]
\FunctionTok{library}\NormalTok{(rpart)}
\FunctionTok{library}\NormalTok{(caret)  }\CommentTok{\# For confusionMatrix}
\end{Highlighting}
\end{Shaded}

\begin{verbatim}
## Warning: пакет 'caret' был собран под R версии 4.4.2
\end{verbatim}

\begin{verbatim}
## Загрузка требуемого пакета: lattice
## 
## Присоединяю пакет: 'caret'
## 
## Следующий объект скрыт от 'package:purrr':
## 
##     lift
\end{verbatim}

\begin{Shaded}
\begin{Highlighting}[]
\CommentTok{\# Load the Titanic dataset}
\NormalTok{titanic }\OtherTok{\textless{}{-}} \FunctionTok{read.csv}\NormalTok{(}\StringTok{"https://raw.githubusercontent.com/datasciencedojo/datasets/master/titanic.csv"}\NormalTok{)}

\CommentTok{\# View the structure of the dataset}
\FunctionTok{str}\NormalTok{(titanic)}
\end{Highlighting}
\end{Shaded}

\begin{verbatim}
## 'data.frame':    891 obs. of  12 variables:
##  $ PassengerId: int  1 2 3 4 5 6 7 8 9 10 ...
##  $ Survived   : int  0 1 1 1 0 0 0 0 1 1 ...
##  $ Pclass     : int  3 1 3 1 3 3 1 3 3 2 ...
##  $ Name       : chr  "Braund, Mr. Owen Harris" "Cumings, Mrs. John Bradley (Florence Briggs Thayer)" "Heikkinen, Miss. Laina" "Futrelle, Mrs. Jacques Heath (Lily May Peel)" ...
##  $ Sex        : chr  "male" "female" "female" "female" ...
##  $ Age        : num  22 38 26 35 35 NA 54 2 27 14 ...
##  $ SibSp      : int  1 1 0 1 0 0 0 3 0 1 ...
##  $ Parch      : int  0 0 0 0 0 0 0 1 2 0 ...
##  $ Ticket     : chr  "A/5 21171" "PC 17599" "STON/O2. 3101282" "113803" ...
##  $ Fare       : num  7.25 71.28 7.92 53.1 8.05 ...
##  $ Cabin      : chr  "" "C85" "" "C123" ...
##  $ Embarked   : chr  "S" "C" "S" "S" ...
\end{verbatim}

\begin{Shaded}
\begin{Highlighting}[]
\CommentTok{\# Preprocess the data}
\NormalTok{titanic }\OtherTok{\textless{}{-}}\NormalTok{ titanic }\SpecialCharTok{\%\textgreater{}\%}
  \FunctionTok{select}\NormalTok{(Survived, Pclass, Sex, Age, SibSp, Parch, Fare) }\SpecialCharTok{\%\textgreater{}\%}
  \FunctionTok{mutate}\NormalTok{(}\AttributeTok{Sex =} \FunctionTok{as.factor}\NormalTok{(Sex),  }\CommentTok{\# Convert Sex to factor}
         \AttributeTok{Survived =} \FunctionTok{as.factor}\NormalTok{(Survived)) }\SpecialCharTok{\%\textgreater{}\%}
  \FunctionTok{drop\_na}\NormalTok{()  }\CommentTok{\# Remove rows with NA values}

\CommentTok{\# Function to generate bootstrap sample}
\NormalTok{bootstrap\_sample }\OtherTok{\textless{}{-}} \ControlFlowTok{function}\NormalTok{(data) \{}
\NormalTok{  n }\OtherTok{\textless{}{-}} \FunctionTok{nrow}\NormalTok{(data)}
\NormalTok{  sample\_indices }\OtherTok{\textless{}{-}} \FunctionTok{sample}\NormalTok{(}\DecValTok{1}\SpecialCharTok{:}\NormalTok{n, }\AttributeTok{size =}\NormalTok{ n, }\AttributeTok{replace =} \ConstantTok{TRUE}\NormalTok{)}
  \FunctionTok{return}\NormalTok{(data[sample\_indices, ])}
\NormalTok{\}}

\CommentTok{\# Function to build a decision tree with feature randomness}
\NormalTok{build\_tree }\OtherTok{\textless{}{-}} \ControlFlowTok{function}\NormalTok{(data, mtry) \{}
  \CommentTok{\# Randomly select a subset of features for each split}
\NormalTok{  features }\OtherTok{\textless{}{-}} \FunctionTok{sample}\NormalTok{(}\FunctionTok{names}\NormalTok{(data)[}\SpecialCharTok{{-}}\DecValTok{1}\NormalTok{], }\AttributeTok{size =}\NormalTok{ mtry)  }\CommentTok{\# Exclude Survived from features}
  
  \CommentTok{\# Build the decision tree using the selected features}
\NormalTok{  tree }\OtherTok{\textless{}{-}} \FunctionTok{rpart}\NormalTok{(Survived }\SpecialCharTok{\textasciitilde{}}\NormalTok{ ., }\AttributeTok{data =}\NormalTok{ data[, }\FunctionTok{c}\NormalTok{(features, }\StringTok{"Survived"}\NormalTok{)])}
  \FunctionTok{return}\NormalTok{(tree)}
\NormalTok{\}}

\CommentTok{\# Function to train Random Forest}
\NormalTok{train\_random\_forest }\OtherTok{\textless{}{-}} \ControlFlowTok{function}\NormalTok{(data, }\AttributeTok{n\_trees =} \DecValTok{50}\NormalTok{, }\AttributeTok{mtry =} \DecValTok{2}\NormalTok{) \{}
\NormalTok{  trees }\OtherTok{\textless{}{-}} \FunctionTok{list}\NormalTok{()}
  
  \ControlFlowTok{for}\NormalTok{ (i }\ControlFlowTok{in} \DecValTok{1}\SpecialCharTok{:}\NormalTok{n\_trees) \{}
    \CommentTok{\# Generate a bootstrap sample of the data}
\NormalTok{    bootstrap\_data }\OtherTok{\textless{}{-}} \FunctionTok{bootstrap\_sample}\NormalTok{(data)}
    
    \CommentTok{\# Build a decision tree on the bootstrap sample}
\NormalTok{    tree }\OtherTok{\textless{}{-}} \FunctionTok{build\_tree}\NormalTok{(bootstrap\_data, mtry)}
    
    \CommentTok{\# Store the tree in the list}
\NormalTok{    trees[[i]] }\OtherTok{\textless{}{-}}\NormalTok{ tree}
\NormalTok{  \}}
  
  \FunctionTok{return}\NormalTok{(trees)}
\NormalTok{\}}

\CommentTok{\# Function to predict using Random Forest}
\NormalTok{predict\_random\_forest }\OtherTok{\textless{}{-}} \ControlFlowTok{function}\NormalTok{(trees, data) \{}
  \CommentTok{\# Get predictions from all trees}
\NormalTok{  predictions }\OtherTok{\textless{}{-}} \FunctionTok{sapply}\NormalTok{(trees, }\ControlFlowTok{function}\NormalTok{(tree) \{}
    \FunctionTok{predict}\NormalTok{(tree, data, }\AttributeTok{type =} \StringTok{"class"}\NormalTok{)}
\NormalTok{  \})}
  
  \CommentTok{\# Apply majority voting}
\NormalTok{  final\_predictions }\OtherTok{\textless{}{-}} \FunctionTok{apply}\NormalTok{(predictions, }\DecValTok{1}\NormalTok{, }\ControlFlowTok{function}\NormalTok{(x) \{}
    \FunctionTok{names}\NormalTok{(}\FunctionTok{sort}\NormalTok{(}\FunctionTok{table}\NormalTok{(x), }\AttributeTok{decreasing =} \ConstantTok{TRUE}\NormalTok{)[}\DecValTok{1}\NormalTok{])}
\NormalTok{  \})}
  
  \FunctionTok{return}\NormalTok{(final\_predictions)}
\NormalTok{\}}

\CommentTok{\# Train Random Forest with 50 trees and 2 features at each split}
\NormalTok{rf\_trees }\OtherTok{\textless{}{-}} \FunctionTok{train\_random\_forest}\NormalTok{(titanic, }\AttributeTok{n\_trees =} \DecValTok{50}\NormalTok{, }\AttributeTok{mtry =} \DecValTok{2}\NormalTok{)}

\CommentTok{\# Predict using the trained Random Forest}
\NormalTok{rf\_preds }\OtherTok{\textless{}{-}} \FunctionTok{predict\_random\_forest}\NormalTok{(rf\_trees, titanic)}

\CommentTok{\# Convert predictions and actual survival to factors with the same levels}
\NormalTok{rf\_preds }\OtherTok{\textless{}{-}} \FunctionTok{factor}\NormalTok{(rf\_preds, }\AttributeTok{levels =} \FunctionTok{levels}\NormalTok{(titanic}\SpecialCharTok{$}\NormalTok{Survived))}
\NormalTok{actual\_survival }\OtherTok{\textless{}{-}} \FunctionTok{factor}\NormalTok{(titanic}\SpecialCharTok{$}\NormalTok{Survived, }\AttributeTok{levels =} \FunctionTok{levels}\NormalTok{(titanic}\SpecialCharTok{$}\NormalTok{Survived))}

\CommentTok{\# Evaluate the model\textquotesingle{}s performance}
\NormalTok{confusion\_matrix }\OtherTok{\textless{}{-}} \FunctionTok{confusionMatrix}\NormalTok{(rf\_preds, actual\_survival)}

\CommentTok{\# Output the confusion matrix and accuracy}
\FunctionTok{print}\NormalTok{(confusion\_matrix)}
\end{Highlighting}
\end{Shaded}

\begin{verbatim}
## Confusion Matrix and Statistics
## 
##           Reference
## Prediction   0   1
##          0 410 129
##          1  14 161
##                                           
##                Accuracy : 0.7997          
##                  95% CI : (0.7685, 0.8285)
##     No Information Rate : 0.5938          
##     P-Value [Acc > NIR] : < 2.2e-16       
##                                           
##                   Kappa : 0.5571          
##                                           
##  Mcnemar's Test P-Value : < 2.2e-16       
##                                           
##             Sensitivity : 0.9670          
##             Specificity : 0.5552          
##          Pos Pred Value : 0.7607          
##          Neg Pred Value : 0.9200          
##              Prevalence : 0.5938          
##          Detection Rate : 0.5742          
##    Detection Prevalence : 0.7549          
##       Balanced Accuracy : 0.7611          
##                                           
##        'Positive' Class : 0               
## 
\end{verbatim}

\begin{Shaded}
\begin{Highlighting}[]
\FunctionTok{library}\NormalTok{(shiny)}
\FunctionTok{library}\NormalTok{(caret)}
\FunctionTok{library}\NormalTok{(rpart)}
\FunctionTok{library}\NormalTok{(dplyr)}
\FunctionTok{library}\NormalTok{(ggplot2)}
\FunctionTok{library}\NormalTok{(shinythemes)}
\end{Highlighting}
\end{Shaded}

\begin{verbatim}
## Warning: пакет 'shinythemes' был собран под R версии 4.4.2
\end{verbatim}

\begin{Shaded}
\begin{Highlighting}[]
\FunctionTok{library}\NormalTok{(shinyalert)}
\end{Highlighting}
\end{Shaded}

\begin{verbatim}
## Warning: пакет 'shinyalert' был собран под R версии 4.4.2
\end{verbatim}

\begin{verbatim}
## 
## Присоединяю пакет: 'shinyalert'
\end{verbatim}

\begin{verbatim}
## Следующий объект скрыт от 'package:shiny':
## 
##     runExample
\end{verbatim}

\begin{Shaded}
\begin{Highlighting}[]
\FunctionTok{library}\NormalTok{(pROC)}
\end{Highlighting}
\end{Shaded}

\begin{verbatim}
## Warning: пакет 'pROC' был собран под R версии 4.4.2
\end{verbatim}

\begin{verbatim}
## Type 'citation("pROC")' for a citation.
\end{verbatim}

\begin{verbatim}
## 
## Присоединяю пакет: 'pROC'
\end{verbatim}

\begin{verbatim}
## Следующие объекты скрыты от 'package:stats':
## 
##     cov, smooth, var
\end{verbatim}

\begin{Shaded}
\begin{Highlighting}[]
\CommentTok{\# Function to generate bootstrap selections}
\NormalTok{bootstrap\_sample }\OtherTok{\textless{}{-}} \ControlFlowTok{function}\NormalTok{(data) \{}
\NormalTok{  n }\OtherTok{\textless{}{-}} \FunctionTok{nrow}\NormalTok{(data)}
\NormalTok{  sample\_indices }\OtherTok{\textless{}{-}} \FunctionTok{sample}\NormalTok{(}\DecValTok{1}\SpecialCharTok{:}\NormalTok{n, }\AttributeTok{size =}\NormalTok{ n, }\AttributeTok{replace =} \ConstantTok{TRUE}\NormalTok{)}
  \FunctionTok{return}\NormalTok{(data[sample\_indices, ])}
\NormalTok{\}}

\CommentTok{\# Function for building a decision tree}
\NormalTok{build\_tree }\OtherTok{\textless{}{-}} \ControlFlowTok{function}\NormalTok{(data, mtry) \{}
\NormalTok{  features }\OtherTok{\textless{}{-}} \FunctionTok{sample}\NormalTok{(}\FunctionTok{names}\NormalTok{(data)[}\SpecialCharTok{{-}}\DecValTok{1}\NormalTok{], }\AttributeTok{size =}\NormalTok{ mtry)}
\NormalTok{  tree }\OtherTok{\textless{}{-}} \FunctionTok{rpart}\NormalTok{(Survived }\SpecialCharTok{\textasciitilde{}}\NormalTok{ ., }\AttributeTok{data =}\NormalTok{ data[, }\FunctionTok{c}\NormalTok{(features, }\StringTok{"Survived"}\NormalTok{)])}
  \FunctionTok{return}\NormalTok{(tree)}
\NormalTok{\}}

\CommentTok{\# Function for learning a random forest}
\NormalTok{train\_random\_forest }\OtherTok{\textless{}{-}} \ControlFlowTok{function}\NormalTok{(data, }\AttributeTok{n\_trees =} \DecValTok{50}\NormalTok{, }\AttributeTok{mtry =} \DecValTok{2}\NormalTok{) \{}
\NormalTok{  trees }\OtherTok{\textless{}{-}} \FunctionTok{list}\NormalTok{()}
  \ControlFlowTok{for}\NormalTok{ (i }\ControlFlowTok{in} \DecValTok{1}\SpecialCharTok{:}\NormalTok{n\_trees) \{}
\NormalTok{    bootstrap\_data }\OtherTok{\textless{}{-}} \FunctionTok{bootstrap\_sample}\NormalTok{(data)}
\NormalTok{    tree }\OtherTok{\textless{}{-}} \FunctionTok{build\_tree}\NormalTok{(bootstrap\_data, mtry)}
\NormalTok{    trees[[i]] }\OtherTok{\textless{}{-}}\NormalTok{ tree}
\NormalTok{  \}}
  \FunctionTok{return}\NormalTok{(trees)}
\NormalTok{\}}

\CommentTok{\# Function for forecasting}
\NormalTok{predict\_random\_forest }\OtherTok{\textless{}{-}} \ControlFlowTok{function}\NormalTok{(trees, data) \{}
\NormalTok{  predictions }\OtherTok{\textless{}{-}} \FunctionTok{sapply}\NormalTok{(trees, }\ControlFlowTok{function}\NormalTok{(tree) \{}
    \FunctionTok{predict}\NormalTok{(tree, data, }\AttributeTok{type =} \StringTok{"class"}\NormalTok{)}
\NormalTok{  \})}
\NormalTok{  final\_predictions }\OtherTok{\textless{}{-}} \FunctionTok{apply}\NormalTok{(predictions, }\DecValTok{1}\NormalTok{, }\ControlFlowTok{function}\NormalTok{(x) \{}
    \FunctionTok{names}\NormalTok{(}\FunctionTok{sort}\NormalTok{(}\FunctionTok{table}\NormalTok{(x), }\AttributeTok{decreasing =} \ConstantTok{TRUE}\NormalTok{)[}\DecValTok{1}\NormalTok{])}
\NormalTok{  \})}
  \FunctionTok{return}\NormalTok{(final\_predictions)}
\NormalTok{\}}

\CommentTok{\# UI Layout}
\NormalTok{ui }\OtherTok{\textless{}{-}} \FunctionTok{fluidPage}\NormalTok{(}
  \AttributeTok{theme =} \FunctionTok{shinytheme}\NormalTok{(}\StringTok{"cerulean"}\NormalTok{),  }\CommentTok{\# Apply a shiny theme for aesthetics}
  
  \FunctionTok{titlePanel}\NormalTok{(}\StringTok{"Random Forest on Titanic Dataset"}\NormalTok{),}
  
  \CommentTok{\# Information section}
  \FunctionTok{wellPanel}\NormalTok{(}
    \FunctionTok{h3}\NormalTok{(}\StringTok{"About this Application"}\NormalTok{),}
    \FunctionTok{p}\NormalTok{(}\StringTok{"This application uses a random forest model to predict survival outcomes on the Titanic dataset. The dataset contains information about passengers, including class, age, sex, and whether they survived or not."}\NormalTok{),}
    \FunctionTok{p}\NormalTok{(}\StringTok{"You can experiment with the number of trees and the number of features per split in the random forest model to see how these affect the accuracy."}\NormalTok{),}
    \FunctionTok{p}\NormalTok{(}\StringTok{"Dataset Source: Data sourced from the Kaggle Titanic dataset."}\NormalTok{),}
    \FunctionTok{br}\NormalTok{(),}
    \FunctionTok{h4}\NormalTok{(}\StringTok{"Inputs:"}\NormalTok{),}
    \FunctionTok{p}\NormalTok{(}\StringTok{"1. The number of trees in the forest (n\_trees)."}\NormalTok{),}
    \FunctionTok{p}\NormalTok{(}\StringTok{"2. The number of features to sample at each split (mtry)."}\NormalTok{),}
    \FunctionTok{p}\NormalTok{(}\StringTok{"3. An action button to trigger training of the model."}\NormalTok{)}
\NormalTok{  ),}
  
  \FunctionTok{sidebarLayout}\NormalTok{(}
    \FunctionTok{sidebarPanel}\NormalTok{(}
      \FunctionTok{sliderInput}\NormalTok{(}\StringTok{"n\_trees"}\NormalTok{, }\StringTok{"Number of Trees:"}\NormalTok{, }\AttributeTok{min =} \DecValTok{10}\NormalTok{, }\AttributeTok{max =} \DecValTok{100}\NormalTok{, }\AttributeTok{value =} \DecValTok{50}\NormalTok{),}
      \FunctionTok{sliderInput}\NormalTok{(}\StringTok{"mtry"}\NormalTok{, }\StringTok{"Number of Features per Split:"}\NormalTok{, }\AttributeTok{min =} \DecValTok{1}\NormalTok{, }\AttributeTok{max =} \DecValTok{7}\NormalTok{, }\AttributeTok{value =} \DecValTok{2}\NormalTok{),}
      \FunctionTok{actionButton}\NormalTok{(}\StringTok{"train"}\NormalTok{, }\StringTok{"Train Model"}\NormalTok{),}
      \FunctionTok{br}\NormalTok{(),}
      \FunctionTok{br}\NormalTok{(),}
      \FunctionTok{helpText}\NormalTok{(}\StringTok{"Use the sliders to adjust the parameters and train the random forest model."}\NormalTok{)}
\NormalTok{    ),}
    
    \FunctionTok{mainPanel}\NormalTok{(}
      \FunctionTok{tabsetPanel}\NormalTok{(}
        \FunctionTok{tabPanel}\NormalTok{(}\StringTok{"Confusion Matrix"}\NormalTok{, }\FunctionTok{verbatimTextOutput}\NormalTok{(}\StringTok{"confMatrix"}\NormalTok{)),}
        \FunctionTok{tabPanel}\NormalTok{(}\StringTok{"Accuracy Plot"}\NormalTok{, }\FunctionTok{plotOutput}\NormalTok{(}\StringTok{"accuracyPlot"}\NormalTok{)),}
        \FunctionTok{tabPanel}\NormalTok{(}\StringTok{"ROC Curve"}\NormalTok{, }\FunctionTok{plotOutput}\NormalTok{(}\StringTok{"rocPlot"}\NormalTok{))}
\NormalTok{      )}
\NormalTok{    )}
\NormalTok{  )}
\NormalTok{)}

\CommentTok{\# Server logic}
\NormalTok{server }\OtherTok{\textless{}{-}} \ControlFlowTok{function}\NormalTok{(input, output) \{}
  
  \FunctionTok{observeEvent}\NormalTok{(input}\SpecialCharTok{$}\NormalTok{train, \{}
    
    \CommentTok{\# Display a loading alert}
    \FunctionTok{shinyalert}\NormalTok{(}\StringTok{"Training Model"}\NormalTok{, }\StringTok{"Please wait while the model is being trained."}\NormalTok{, }\AttributeTok{type =} \StringTok{"info"}\NormalTok{)}
    
    \CommentTok{\# Upload Titanic dataset}
\NormalTok{    titanic }\OtherTok{\textless{}{-}} \FunctionTok{read.csv}\NormalTok{(}\StringTok{"https://raw.githubusercontent.com/datasciencedojo/datasets/master/titanic.csv"}\NormalTok{)}
    
    \CommentTok{\# Data preprocessing}
\NormalTok{    titanic }\OtherTok{\textless{}{-}}\NormalTok{ titanic }\SpecialCharTok{\%\textgreater{}\%}
      \FunctionTok{select}\NormalTok{(Survived, Pclass, Sex, Age, SibSp, Parch, Fare) }\SpecialCharTok{\%\textgreater{}\%}
      \FunctionTok{mutate}\NormalTok{(}\AttributeTok{Sex =} \FunctionTok{as.factor}\NormalTok{(Sex), }\AttributeTok{Survived =} \FunctionTok{as.factor}\NormalTok{(Survived)) }\SpecialCharTok{\%\textgreater{}\%}
      \FunctionTok{drop\_na}\NormalTok{()}
    
    \CommentTok{\# Model training}
\NormalTok{    rf\_trees }\OtherTok{\textless{}{-}} \FunctionTok{train\_random\_forest}\NormalTok{(titanic, }\AttributeTok{n\_trees =}\NormalTok{ input}\SpecialCharTok{$}\NormalTok{n\_trees, }\AttributeTok{mtry =}\NormalTok{ input}\SpecialCharTok{$}\NormalTok{mtry)}
\NormalTok{    rf\_preds }\OtherTok{\textless{}{-}} \FunctionTok{predict\_random\_forest}\NormalTok{(rf\_trees, titanic)}
    
    \CommentTok{\# Bring predictions to factors with the same levels}
\NormalTok{    rf\_preds }\OtherTok{\textless{}{-}} \FunctionTok{factor}\NormalTok{(rf\_preds, }\AttributeTok{levels =} \FunctionTok{levels}\NormalTok{(titanic}\SpecialCharTok{$}\NormalTok{Survived))}
    
    \CommentTok{\# Creating a confusion matrix}
\NormalTok{    confusion\_matrix }\OtherTok{\textless{}{-}} \FunctionTok{confusionMatrix}\NormalTok{(rf\_preds, titanic}\SpecialCharTok{$}\NormalTok{Survived)}
    
    \CommentTok{\# Calculate model accuracy}
\NormalTok{    accuracy }\OtherTok{\textless{}{-}}\NormalTok{ confusion\_matrix}\SpecialCharTok{$}\NormalTok{overall[}\StringTok{"Accuracy"}\NormalTok{]}
    
    \CommentTok{\# Output the confusion matrix}
\NormalTok{    output}\SpecialCharTok{$}\NormalTok{confMatrix }\OtherTok{\textless{}{-}} \FunctionTok{renderPrint}\NormalTok{(\{ confusion\_matrix \})}
    
    \CommentTok{\# Output accuracy plot (bar plot of confusion matrix)}
\NormalTok{    output}\SpecialCharTok{$}\NormalTok{accuracyPlot }\OtherTok{\textless{}{-}} \FunctionTok{renderPlot}\NormalTok{(\{}
      \FunctionTok{barplot}\NormalTok{(confusion\_matrix}\SpecialCharTok{$}\NormalTok{table, }\AttributeTok{main =} \StringTok{"Confusion Matrix"}\NormalTok{, }\AttributeTok{col =} \FunctionTok{c}\NormalTok{(}\StringTok{"blue"}\NormalTok{, }\StringTok{"red"}\NormalTok{), }\AttributeTok{beside =} \ConstantTok{TRUE}\NormalTok{)}
\NormalTok{    \})}
    
    \CommentTok{\# Calculate and plot the ROC curve}
\NormalTok{    roc\_curve }\OtherTok{\textless{}{-}} \FunctionTok{roc}\NormalTok{(titanic}\SpecialCharTok{$}\NormalTok{Survived, }\FunctionTok{as.numeric}\NormalTok{(rf\_preds))}
\NormalTok{    output}\SpecialCharTok{$}\NormalTok{rocPlot }\OtherTok{\textless{}{-}} \FunctionTok{renderPlot}\NormalTok{(\{}
      \FunctionTok{plot}\NormalTok{(roc\_curve, }\AttributeTok{main =} \StringTok{"ROC Curve"}\NormalTok{, }\AttributeTok{col =} \StringTok{"blue"}\NormalTok{, }\AttributeTok{lwd =} \DecValTok{2}\NormalTok{)}
      \FunctionTok{abline}\NormalTok{(}\AttributeTok{a =} \DecValTok{0}\NormalTok{, }\AttributeTok{b =} \DecValTok{1}\NormalTok{, }\AttributeTok{col =} \StringTok{"red"}\NormalTok{, }\AttributeTok{lty =} \DecValTok{2}\NormalTok{)}
\NormalTok{    \})}
    
    \CommentTok{\# Display success message}
    \FunctionTok{shinyalert}\NormalTok{(}\StringTok{"Training Complete"}\NormalTok{, }\FunctionTok{paste}\NormalTok{(}\StringTok{"Model trained successfully. Accuracy: "}\NormalTok{, }\FunctionTok{round}\NormalTok{(accuracy, }\DecValTok{4}\NormalTok{) }\SpecialCharTok{*} \DecValTok{100}\NormalTok{, }\StringTok{"\%"}\NormalTok{), }\AttributeTok{type =} \StringTok{"success"}\NormalTok{)}
\NormalTok{  \})}
\NormalTok{\}}

\CommentTok{\# Run the application }
\FunctionTok{shinyApp}\NormalTok{(}\AttributeTok{ui =}\NormalTok{ ui, }\AttributeTok{server =}\NormalTok{ server)}
\end{Highlighting}
\end{Shaded}


\end{document}
